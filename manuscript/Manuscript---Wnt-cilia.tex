% Options for packages loaded elsewhere
\PassOptionsToPackage{unicode}{hyperref}
\PassOptionsToPackage{hyphens}{url}
\PassOptionsToPackage{dvipsnames,svgnames,x11names}{xcolor}
%
\documentclass[
  11pt,
]{article}

\usepackage{amsmath,amssymb}
\usepackage{setspace}
\usepackage{iftex}
\ifPDFTeX
  \usepackage[T1]{fontenc}
  \usepackage[utf8]{inputenc}
  \usepackage{textcomp} % provide euro and other symbols
\else % if luatex or xetex
  \usepackage{unicode-math}
  \defaultfontfeatures{Scale=MatchLowercase}
  \defaultfontfeatures[\rmfamily]{Ligatures=TeX,Scale=1}
\fi
\usepackage{lmodern}
\ifPDFTeX\else  
    % xetex/luatex font selection
  \setmainfont[]{Arial}
  \setmonofont[]{Courier New}
\fi
% Use upquote if available, for straight quotes in verbatim environments
\IfFileExists{upquote.sty}{\usepackage{upquote}}{}
\IfFileExists{microtype.sty}{% use microtype if available
  \usepackage[]{microtype}
  \UseMicrotypeSet[protrusion]{basicmath} % disable protrusion for tt fonts
}{}
\makeatletter
\@ifundefined{KOMAClassName}{% if non-KOMA class
  \IfFileExists{parskip.sty}{%
    \usepackage{parskip}
  }{% else
    \setlength{\parindent}{0pt}
    \setlength{\parskip}{6pt plus 2pt minus 1pt}}
}{% if KOMA class
  \KOMAoptions{parskip=half}}
\makeatother
\usepackage{xcolor}
\usepackage[lmargin=0.8in,rmargin=0.8in,tmargin=0.8in,bmargin=0.8in]{geometry}
\ifLuaTeX
  \usepackage{luacolor}
  \usepackage[soul]{lua-ul}
\else
  \usepackage{soul}
  
\fi
\setlength{\emergencystretch}{3em} % prevent overfull lines
\setcounter{secnumdepth}{-\maxdimen} % remove section numbering
% Make \paragraph and \subparagraph free-standing
\ifx\paragraph\undefined\else
  \let\oldparagraph\paragraph
  \renewcommand{\paragraph}[1]{\oldparagraph{#1}\mbox{}}
\fi
\ifx\subparagraph\undefined\else
  \let\oldsubparagraph\subparagraph
  \renewcommand{\subparagraph}[1]{\oldsubparagraph{#1}\mbox{}}
\fi

\usepackage{color}
\usepackage{fancyvrb}
\newcommand{\VerbBar}{|}
\newcommand{\VERB}{\Verb[commandchars=\\\{\}]}
\DefineVerbatimEnvironment{Highlighting}{Verbatim}{commandchars=\\\{\}}
% Add ',fontsize=\small' for more characters per line
\usepackage{framed}
\definecolor{shadecolor}{RGB}{241,243,245}
\newenvironment{Shaded}{\begin{snugshade}}{\end{snugshade}}
\newcommand{\AlertTok}[1]{\textcolor[rgb]{0.68,0.00,0.00}{#1}}
\newcommand{\AnnotationTok}[1]{\textcolor[rgb]{0.37,0.37,0.37}{#1}}
\newcommand{\AttributeTok}[1]{\textcolor[rgb]{0.40,0.45,0.13}{#1}}
\newcommand{\BaseNTok}[1]{\textcolor[rgb]{0.68,0.00,0.00}{#1}}
\newcommand{\BuiltInTok}[1]{\textcolor[rgb]{0.00,0.23,0.31}{#1}}
\newcommand{\CharTok}[1]{\textcolor[rgb]{0.13,0.47,0.30}{#1}}
\newcommand{\CommentTok}[1]{\textcolor[rgb]{0.37,0.37,0.37}{#1}}
\newcommand{\CommentVarTok}[1]{\textcolor[rgb]{0.37,0.37,0.37}{\textit{#1}}}
\newcommand{\ConstantTok}[1]{\textcolor[rgb]{0.56,0.35,0.01}{#1}}
\newcommand{\ControlFlowTok}[1]{\textcolor[rgb]{0.00,0.23,0.31}{#1}}
\newcommand{\DataTypeTok}[1]{\textcolor[rgb]{0.68,0.00,0.00}{#1}}
\newcommand{\DecValTok}[1]{\textcolor[rgb]{0.68,0.00,0.00}{#1}}
\newcommand{\DocumentationTok}[1]{\textcolor[rgb]{0.37,0.37,0.37}{\textit{#1}}}
\newcommand{\ErrorTok}[1]{\textcolor[rgb]{0.68,0.00,0.00}{#1}}
\newcommand{\ExtensionTok}[1]{\textcolor[rgb]{0.00,0.23,0.31}{#1}}
\newcommand{\FloatTok}[1]{\textcolor[rgb]{0.68,0.00,0.00}{#1}}
\newcommand{\FunctionTok}[1]{\textcolor[rgb]{0.28,0.35,0.67}{#1}}
\newcommand{\ImportTok}[1]{\textcolor[rgb]{0.00,0.46,0.62}{#1}}
\newcommand{\InformationTok}[1]{\textcolor[rgb]{0.37,0.37,0.37}{#1}}
\newcommand{\KeywordTok}[1]{\textcolor[rgb]{0.00,0.23,0.31}{#1}}
\newcommand{\NormalTok}[1]{\textcolor[rgb]{0.00,0.23,0.31}{#1}}
\newcommand{\OperatorTok}[1]{\textcolor[rgb]{0.37,0.37,0.37}{#1}}
\newcommand{\OtherTok}[1]{\textcolor[rgb]{0.00,0.23,0.31}{#1}}
\newcommand{\PreprocessorTok}[1]{\textcolor[rgb]{0.68,0.00,0.00}{#1}}
\newcommand{\RegionMarkerTok}[1]{\textcolor[rgb]{0.00,0.23,0.31}{#1}}
\newcommand{\SpecialCharTok}[1]{\textcolor[rgb]{0.37,0.37,0.37}{#1}}
\newcommand{\SpecialStringTok}[1]{\textcolor[rgb]{0.13,0.47,0.30}{#1}}
\newcommand{\StringTok}[1]{\textcolor[rgb]{0.13,0.47,0.30}{#1}}
\newcommand{\VariableTok}[1]{\textcolor[rgb]{0.07,0.07,0.07}{#1}}
\newcommand{\VerbatimStringTok}[1]{\textcolor[rgb]{0.13,0.47,0.30}{#1}}
\newcommand{\WarningTok}[1]{\textcolor[rgb]{0.37,0.37,0.37}{\textit{#1}}}

\providecommand{\tightlist}{%
  \setlength{\itemsep}{0pt}\setlength{\parskip}{0pt}}\usepackage{longtable,booktabs,array}
\usepackage{calc} % for calculating minipage widths
% Correct order of tables after \paragraph or \subparagraph
\usepackage{etoolbox}
\makeatletter
\patchcmd\longtable{\par}{\if@noskipsec\mbox{}\fi\par}{}{}
\makeatother
% Allow footnotes in longtable head/foot
\IfFileExists{footnotehyper.sty}{\usepackage{footnotehyper}}{\usepackage{footnote}}
\makesavenoteenv{longtable}
\usepackage{graphicx}
\makeatletter
\def\maxwidth{\ifdim\Gin@nat@width>\linewidth\linewidth\else\Gin@nat@width\fi}
\def\maxheight{\ifdim\Gin@nat@height>\textheight\textheight\else\Gin@nat@height\fi}
\makeatother
% Scale images if necessary, so that they will not overflow the page
% margins by default, and it is still possible to overwrite the defaults
% using explicit options in \includegraphics[width, height, ...]{}
\setkeys{Gin}{width=\maxwidth,height=\maxheight,keepaspectratio}
% Set default figure placement to htbp
\makeatletter
\def\fps@figure{htbp}
\makeatother
% definitions for citeproc citations
\NewDocumentCommand\citeproctext{}{}
\NewDocumentCommand\citeproc{mm}{%
  \begingroup\def\citeproctext{#2}\cite{#1}\endgroup}
\makeatletter
 % allow citations to break across lines
 \let\@cite@ofmt\@firstofone
 % avoid brackets around text for \cite:
 \def\@biblabel#1{}
 \def\@cite#1#2{{#1\if@tempswa , #2\fi}}
\makeatother
\newlength{\cslhangindent}
\setlength{\cslhangindent}{1.5em}
\newlength{\csllabelwidth}
\setlength{\csllabelwidth}{3em}
\newenvironment{CSLReferences}[2] % #1 hanging-indent, #2 entry-spacing
 {\begin{list}{}{%
  \setlength{\itemindent}{0pt}
  \setlength{\leftmargin}{0pt}
  \setlength{\parsep}{0pt}
  % turn on hanging indent if param 1 is 1
  \ifodd #1
   \setlength{\leftmargin}{\cslhangindent}
   \setlength{\itemindent}{-1\cslhangindent}
  \fi
  % set entry spacing
  \setlength{\itemsep}{#2\baselineskip}}}
 {\end{list}}
\usepackage{calc}
\newcommand{\CSLBlock}[1]{\hfill\break\parbox[t]{\linewidth}{\strut\ignorespaces#1\strut}}
\newcommand{\CSLLeftMargin}[1]{\parbox[t]{\csllabelwidth}{\strut#1\strut}}
\newcommand{\CSLRightInline}[1]{\parbox[t]{\linewidth - \csllabelwidth}{\strut#1\strut}}
\newcommand{\CSLIndent}[1]{\hspace{\cslhangindent}#1}

\usepackage{float}
\usepackage{tabularray}
\usepackage[normalem]{ulem}
\usepackage{graphicx}
\UseTblrLibrary{booktabs}
\UseTblrLibrary{siunitx}
\NewTableCommand{\tinytableDefineColor}[3]{\definecolor{#1}{#2}{#3}}
\newcommand{\tinytableTabularrayUnderline}[1]{\underline{#1}}
\newcommand{\tinytableTabularrayStrikeout}[1]{\sout{#1}}
\usepackage{lineno}
\makeatletter
\@ifpackageloaded{caption}{}{\usepackage{caption}}
\AtBeginDocument{%
\ifdefined\contentsname
  \renewcommand*\contentsname{Table of contents}
\else
  \newcommand\contentsname{Table of contents}
\fi
\ifdefined\listfigurename
  \renewcommand*\listfigurename{List of Figures}
\else
  \newcommand\listfigurename{List of Figures}
\fi
\ifdefined\listtablename
  \renewcommand*\listtablename{List of Tables}
\else
  \newcommand\listtablename{List of Tables}
\fi
\ifdefined\figurename
  \renewcommand*\figurename{Figure}
\else
  \newcommand\figurename{Figure}
\fi
\ifdefined\tablename
  \renewcommand*\tablename{Table}
\else
  \newcommand\tablename{Table}
\fi
}
\@ifpackageloaded{float}{}{\usepackage{float}}
\floatstyle{ruled}
\@ifundefined{c@chapter}{\newfloat{codelisting}{h}{lop}}{\newfloat{codelisting}{h}{lop}[chapter]}
\floatname{codelisting}{Listing}
\newcommand*\listoflistings{\listof{codelisting}{List of Listings}}
\makeatother
\makeatletter
\makeatother
\makeatletter
\@ifpackageloaded{caption}{}{\usepackage{caption}}
\@ifpackageloaded{subcaption}{}{\usepackage{subcaption}}
\makeatother
\ifLuaTeX
  \usepackage{selnolig}  % disable illegal ligatures
\fi
\usepackage{bookmark}

\IfFileExists{xurl.sty}{\usepackage{xurl}}{} % add URL line breaks if available
\urlstyle{same} % disable monospaced font for URLs
\hypersetup{
  pdftitle={Wnt/LRP6 signaling activation impairs ciliogenesis via suppression of OFD1 degradation via mTOR-dependent autophagy},
  colorlinks=true,
  linkcolor={blue},
  filecolor={Maroon},
  citecolor={Blue},
  urlcolor={Blue},
  pdfcreator={LaTeX via pandoc}}

\title{Wnt/LRP6 signaling activation impairs ciliogenesis via
suppression of OFD1 degradation via mTOR-dependent autophagy}
\usepackage{etoolbox}
\makeatletter
\providecommand{\subtitle}[1]{% add subtitle to \maketitle
  \apptocmd{\@title}{\par {\large #1 \par}}{}{}
}
\makeatother
\subtitle{\raggedright

Cheng Yuan 1\textsuperscript{1,2} ,Bahtiyar Kurtulmus
2\textsuperscript{1,2}, Sergio Acebron 3\textsuperscript{3}, Gislene
Pereira 4\textsuperscript{4,*} \newline\textsuperscript{1}Heidelberg
University, Centre for Organismal Studies (COS), 69120 Heidelberg,
Germany \newline\textsuperscript{2}Affiliation 2
\newline\textsuperscript{3}Affiliation 3
\newline\textsuperscript{4}Affiliation 4\newline
\textsuperscript{*}Correspondence: xy@cos.uni-heidelberg.de}
\author{}
\date{}

\begin{document}
\maketitle

\linenumbers

\renewcommand*\contentsname{Table of contents}
{
\hypersetup{linkcolor=}
\setcounter{tocdepth}{3}
\tableofcontents
}
\setstretch{1.2}
\section{Abstract}\label{abstract}

\ul{This} is an R \textbf{Markdown} \emph{document.}
\textsuperscript{Markdown} is a simple formatting syntax for authoring
HTML, PDF, and MS Word documents. For more details on using R Markdown
see \url{http://rmarkdown.rstudio.com}.

\textbf{text in bold} \emph{italic} \ul{underline}

\section{Introduction}\label{introduction}

You can add references either by referring to their id in the .bib file
e.g., (Marinković et al., 2019), or by switching to the visual editor
(Cogwheel in the .Rmd menu -\textgreater{} Use Visual Editor). {[}Jokura
et al. (2023){]}(Jokura et al., 2023)(Jacobs and Ryu, 2023)

It is now a test to edit the text and see how the changes show up on
GitHub.

Test of git show.

In the visual editor mode, go to `Insert' -\textgreater{} @ Citation

You can select a Zotero library, PubMed, CrossRef etc. and insert the
citations. (Jacobs and Ryu, 2023)

The easiest way is to use the command line:

\begin{Shaded}
\begin{Highlighting}[]
\ExtensionTok{curl} \AttributeTok{{-}LH} \StringTok{"Accept: application/x{-}bibtex"}\NormalTok{ https://doi.org/10.7554/eLife.91258.1 }\OperatorTok{\textgreater{}\textgreater{}}\NormalTok{ references.bib}
\end{Highlighting}
\end{Shaded}

\emph{Platynereis dumerilii} is a marine annelid\ldots{} (Ozpolat et
al., 2021)

The references are stored in manuscript/references.bib (need to be
defined in the Yaml header). This file will automatically updated when
you insert a new reference through the Visual editor \textgreater{}
Insert \textgreater{} Citations.

In this documents, references will be formatted in the style of eLife.
This is defined in the Yaml header under: csl: elife.csl. The elife.csl
file is saved in the /manuscript folder.

If you would like to use a different citation format, download the
respective .csl file (e.g., from the Zotero style repository
\url{https://www.zotero.org/styles}), save it in the /manuscript folder
of the project and change the Yaml to csl: your\_favourite\_journal.csl.

\section{Results}\label{results}

\subsubsection{Wnt/LRP6 signaling stimulation decreases cilia
formation}\label{wntlrp6-signaling-stimulation-decreases-cilia-formation}

To investigate the influence of WNT activation upon cilia formation, we
used human retinal pigment epithelial (RPE1) cells, which undergo
ciliation in a time-dependent manner upon serum starvation. To confirm
the responsiveness of our RPE1 cell line to the WNT ligand Wnt3a and
establish the timing of Wnt3a treatment, we constructed stable cells
carrying the established WNT reporter 7xTGC-GFP (REF). The addition of
Wnt3a-conditioned media (Wnt3a-CM), but not control-CM (Co-CM), led to a
timely increase in the number of cells with high GFP signals, clearly
visible after 6 hours of Wnt3a-CM addition (Fig. S1A). Furthermore, we
observed increased β-catenin protein levels as a consequence of
β-catenin stabilization (Fig. S1B). Based on these analyses, RPE1 cells
were treated twice (6 hours prior to serum starvation and at the time of
serum starvation) with Co-CM or Wnt3a-CM to maintain WNT activation
(Fig. 1A). In comparison to control, Wnt3a-CM treatment significant
reducted ciliation in RPE1 cells even after prolonged incubation
(Fig.1B, S1C). Importantly, the analysis of DNA-content by FACS and
phosphorylated Rb protein (a marker for proliferating cells, REF)
indicated that this negative effect on ciliogenesis was not due to
inability of cells to arrest at the G1/G0 phase of the cell cycle upon
serum starvation (Fig. 1C, S1D). The negative effect of Wnt activation
on ciliogenesis was not restricted to RPE1 cells, as a similar effect
was observed in mouse fibroblasts (NIH3T3, Fig. S1E, S1F) and murine
inner medullar kidney tubular (IMCD3) cells (Fig. S1E, S1F). Together,
these data show that activation of Wnt signaling decreases the ability
of human RPE1, mouse NIH3T3 and IMCD3 cells to ciliate. WNT ligands,
such as Wnt3a and Wnt1, signal via heterodimerisation of the WNT
receptor frizzled (FZD) and co-receptors LRP5/LRP6. This activation step
is counteracted by the Dickkopf-related protein 1 (DKK1), which inhibits
the association of WNT ligand to LRP receptors. To address whether the
effect of WNT on ciliogenesis requires FZD-LRP5/6, we first made use of
a recombinant surrogate WNT agonist, which induces heterodimerisation of
FZD and LRP5 or LRP6 and hence WNT activation (Fig S1G). Similar to
Wnt3a-CM, recombinant Wnt surrogate significantly diminished the
percentage of ciliated cells in RPE1 cells (Fig. 1D). Furthermore, DKK1
counteracted the negative effect of Wnt1 ligand on ciliogenesis, as the
co-expression of Wnt1 and DKK1 was able to rescue ciliogenesis in Wnt1
expressing cells (Fig. 1E). This data together indicate the involvement
of the WNT-LRP pathway in ciliogenesis.

\subsubsection{Wnt/LRP-activation inhibits cilia formation independently
of
TCF7}\label{wntlrp-activation-inhibits-cilia-formation-independently-of-tcf7}

Activation of Wnt/LRP6 pathway leads to stabilisation of cytoplasmic
β-catenin that subsequently enters the nucleus, where it activates gene
expression via binding to Lymphoid enhancer-binding factor (Lef)/T-cell
factor (TCF) transcription factors. To test whether Wnt activation
influences ciliogenesis via TCF/LEF dependent transcriptional
regulation, we investigated whether ciliogenesis required TCF7 (Fig.
2A-C). In the large majority of RPE1 cells (\textgreater99 \%), TCF7
entered the nucleus upon Wnt3a addition (Fig. 2A). Furthermore, the
levels of TCF7 and β-catenin protein increased upon Wnt3a-CM but not
Co-CM addition in mock-depleted cells (Fig. 2A and 2B, siLUC),
confirming responsiveness to Wnt activation in RPE1 cells. We next
depleted TCF7 using small interference RNAs (siRNAs) before exposing the
cells to Co- or Wnt3a-CM. Additional markers for WNT activation included
DVL2 hyperphosphorylation, which is induced via an LRP5/6 independent
mechanism independently of β-catenin dependent transcription
(10.1128/MCB.24.11.4757-4768.2004). DVL2 became hyper-phosphorylated
upon addition of Wnt3a-CM in mock-siRNA or TCF7-siRNA treated cells,
confirming Wnt activation in both conditions. Interestingly, β-catenin
protein levels decreased in TCF7-siRNA treated cells indicating
activates β-catenin protein in a TCF7-dependent manner (Fig. 2B) (REFS).
However, the analysis of ciliogenesis indicated that Wnt3a-CM led to
cilia loss independently of the presence of TCF7 (Fig. 2C). In
conclusion, these data indicate that the negative effect of Wnt
activation upon cilia formation does not require TCF7.

\subsubsection{Wnt-LRP activation decreases the recruitment of Rab8a to
basal
bodies}\label{wnt-lrp-activation-decreases-the-recruitment-of-rab8a-to-basal-bodies}

GSK3β was shown to promote ciliary membrane extension in a
Rab8a-dependent manner (REF Zhang). Indeed, the inhibition of GSK3 in
RPE1 cells with either BIO or CHIR99021 significantly suppressed cilia
formation (Fig. X) and the accumulation of Rab8a around centrosomes
(Fig. X). Interestingly, few cells were still able to form cilia upon
GSK3 inhibition. In these cells, cilia were significantly longer in
comparison to WT cells, confirming GSK3's influence on cilia biogenesis
(REF for longer cilia).

As GSK3β decreases upon WNT activation, we asked whether Rab8a-vesicular
trafficking to the centrosome was also impaired in Wnt3a-treated cells.
To investigate this, we used an established RPE1 cell line stably
expressing GFP-Rab8a (Ref Kerstin). Whereas GFP-Rab8a accumulated around
centrosomes shortly after serum starvation in control cells, the number
of Rab8a-positive centrosomes significantly reduced in Wnt3a-CM treated
samples (Fig. X). Since Rab8a is transported to the centrosome in a
microtubule-dependent manner (Ref), we also analyzed the localization of
the centriolar satellite protein PCM1, which accumulates around
centrosomes via microtubule-based transport (REF), under the same
conditions. The levels of centrosomal PCM1 did not differ between Co-CM
and Wnt3a-CM treated cells (Fig. SX), implying that
microtubule-dependent trafficking of proteins to the centrosome is not
generally disturbed.

Rab8a-vesicle docking at mother centrioles requires distal appendage
proteins (REF Kerstin, Sillibourne, Tsou). To investigate whether Wnt
activation negatively influenced appendage formation, we compared the
levels of appendage proteins at the basal body in Co-CM and Wnt3a-CM
treated cells. No significant decrease in protein levels was observed
for CEP164, TTBK2, CEP83, CEP89/CEP123 or CHIBBY (distal appendage
proteins, Fig. 2A, Fig. SX, X -- MOVE all negative data to supplement).
The levels of the subdistal component ODF2 also remained unchanged upon
Wnt3a-CM treatment (Fig. X). Interestingly, the levels of CEP164 and
CHIBBY were significantly decreased in GSK3-inhibited RPE1 cells
compared to untreated cells (Fig. X), indicating a role of GSK3 kinase
in appendage integrity.

Data with DVL overexpression and signalosome formation as competitor of
Rab8 here?

\subsubsection{Inhibition of mTOR signaling rescues cilia formation in
Wnt activated
cells}\label{inhibition-of-mtor-signaling-rescues-cilia-formation-in-wnt-activated-cells}

Autophagy was shown to play an important role in ciliogenesis through
degradation of cilia inhibitory proteins. As Wnt signaling was reported
to activate mTOR, which in turn reduces autophagy (Nazio paper), we
asked whether Wnt activation influences ciliogenesis through mTOR. If
this were the case, we would expect higher mTOR activity in cells
treated with Wnt3a-CM. The phosphorylation levels of the mTOR substrate,
the ribosomal protein S6 kinase, S6K, markedly increased in Wnt3a-CM
(Fig. X, lane 3) in comparison to control (Fig X, lane 1). This effect
was due to mTOR activity, as S6K phosphorylation was inhibited by the
mTOR inhibitor, rapamycin (Fig X., lanes 2 and 4). We thus concluded
that mTOR activity is higher in Wnt activated samples. To test whether
increased mTOR activity was related to loss of cilia upon Wnt
activation, we exposed Co-CM and Wnt3a-CM treated samples to a short
treatment with rapamycin (Fig. X). This treatment significantly
increased ciliation in Wnt3a-CM but not in Co-CM treated cells (Fig. X),
confirming the contribution of mTOR to cilia loss. Together, our data
suggest that WNT activation inpairs cilia formation via the
mTOR-autophagy pathway.

\subsubsection{Wnt signaling activation impairs the removal of OFD1 from
centriolar
satellites}\label{wnt-signaling-activation-impairs-the-removal-of-ofd1-from-centriolar-satellites}

One autophagy target involved in cilia formation is the
ciliopathy-related protein OFD1. OFD1 is present at both, centrioles and
centriolar satellites. Centriolar satellites are macromolecular protein
complexes containing centrosome/cilia components that are distributed
within the cytoplasm and enriched at the centrosome. At centrioles, OFD1
is essential for centriole biogenesis and ciliogenesis, whereas the
satellite pool of OFD1 suppresses cilia formation and is degraded by the
autophagic pathway. To investigate whether WNT activation influences
OFD1 degradation, we quantified the levels of centriolar satellite OFD1
at the pericentrosomal area in the presence or absence of Wnt3a. We
observed significantly higher levels of OFD1 at centriolar satellites in
cells treated with Wnt3a in comparison to control samples (Figure XC and
D), implying that OFD1 degradation is impaired upon Wnt activation.

We next reasoned that if the loss of cilia formation upon Wnt activation
was due to the presence of OFD1, it ectopic degradation should rescue
ciliogenesis. For this, we used siRNA-treatment to reduce the levels of
OFD1 prior to cilia induction and Wnt3a-treatment. The depletion of ODF1
significantly increased the percentage of ciliated cells in Wnt3a-CM
treated samples (Fig. XE). This suggest that Wnt-activation impairs
ciliogenesis by impairing OFD1 centriole satellite degradation.

You can add your figures into the rendered document. We saved the
figures into /manuscript/figures or /manuscript/figure\_supplements and
can insert them from there. We use knitr::include\_graphics for this.
The title and legend can also be edited, as will as the width of the
output figure. Test comment behaviour:

\begin{figure}[H]

{\centering \includegraphics[width=1\textwidth,height=\textheight]{figures/Figure1.png}

}

\caption{\textbf{Figure 1. A figure} (A) A nice picture. (B) legend. (C)
(D)}

\end{figure}%

\begin{figure}[H]

{\centering \includegraphics[width=1\textwidth,height=\textheight]{figures/Figure_complex.png}

}

\caption{\textbf{Figure 1. Our nice figure from yesterday} (A) A nice
picture. (B) legend. (C) (D)}

\end{figure}%

\subsection{Equations}\label{equations}

Equations can also be inserted, Insert -\textgreater{} Display Math:

\[
\bar{X} = \frac{\sum_{i=1}^{n} x_{i}}{n}
\]

\hfill\break

\subsection{Sourcing code and working with
variable}\label{sourcing-code-and-working-with-variable}

The mean value of Nanog expression was 0.0909 indicating that Nanog is
downregulated. The `analysis/scripts/statistics\_for\_paper.R' script is
sourced and it runs but the output is not included in the knitted
output. But we can access the variables defined in the sourced script
simply by adding ` r var\_name ` between ` backticks, in this case
max\_PRC value is 21 (now this number comes from our sourced script).

If we update the data, the script can recalculate the variable we want
to refer to in the text and update the number.

\subsection{Acknowledgements}\label{acknowledgements}

We would like to thank the Jekely lab for the R project template
(\url{https://github.com/JekelyLab/new_paper_template}) we used to write
this paper. This work was funded by \ldots{}

\subsection{Materials and Methods}\label{materials-and-methods}

You can insert tables from source data, such as .csv or Excel files and
render them in html with the tinytable package.

Alternatively, you can use the Markdown grid table format. For more
complex tables, you can use the
\href{https://www.tablesgenerator.com/markdown_tables}{tablesgenerator}
online grid table editor/converter (e.g.~converts csv or excel files).

The output may differ between html and pdf, for most consistent results
use the grid table format described
\href{https://quarto.org/docs/authoring/tables.html}{here}.

\textbf{Key Resources Table}

\begin{verbatim}
Warning: package 'tinytable' was built under R version 4.4.1
\end{verbatim}

\begin{verbatim}
Warning: The `placement` argument in `tt()` is deprecated. Please use this
instead: `theme_tt(table, 'placement')`
\end{verbatim}

\begin{table}[H]
\centering
\begin{tblr}[         %% tabularray outer open
]                     %% tabularray outer close
{                     %% tabularray inner open
width={1\linewidth},
colspec={X[]X[]X[]X[]X[]},
hlines, vlines,
}                     %% tabularray inner close
Reagent type (species) or resource & Designation & Source or reference & Identifiers & Additional information \\
biological sample (N. vectensis)     & larval, juvenile and adult N. vectensis        & Specimens obtained form the Marine Invertebrate Culture Unit of the University of Exeter & N/A      & NA                                                                                                               \\
biological sample (cDNA)             & cDNA obtained from N. vectensis                & this study                                                                               & N/A      & RNA extracted with Trizol and cDNA synthesized with cDNA synthesis kit according to manufacturers recommendation \\
biological sample (peptide extract)  & peptide extracts obtained from N. vectensis    & this study                                                                               & N/A      & Peptides extracted from N. vectensis according to protocol explained in Material and Methods                     \\
genetic reagent (cDNA synthesis)     & SuperScript™ III First-Strand Synthesis System & Invitrogen (from ThermoFisher)                                                           & 18080051 & NA                                                                                                               \\
genetic reagent (Polymerase)         & Q5® Hot Start High-Fidelity DNA Polymerase     & New England Biolabs                                                                      & M0493L   & NA                                                                                                               \\
genetic reagent (DNA assembly)       & NEBuilder® HiFi DNA Assembly Master Mix        & New England Biolabs                                                                      & E2621L   & NA                                                                                                               \\
genetic reagent (restriction enzyme) & EcoRV restriction enzyme                       & New England Biolabs                                                                      & R3195L   & NA                                                                                                               \\
genetic reagent (restriction enzyme) & Afl2 restriction enzyme                        & New England Biolabs                                                                      & R0520L   & NA                                                                                                               \\
\end{tblr}
\end{table}

\begin{longtable}[]{@{}
  >{\raggedright\arraybackslash}p{(\columnwidth - 8\tabcolsep) * \real{0.1000}}
  >{\raggedright\arraybackslash}p{(\columnwidth - 8\tabcolsep) * \real{0.2000}}
  >{\raggedright\arraybackslash}p{(\columnwidth - 8\tabcolsep) * \real{0.2000}}
  >{\raggedright\arraybackslash}p{(\columnwidth - 8\tabcolsep) * \real{0.2000}}
  >{\raggedright\arraybackslash}p{(\columnwidth - 8\tabcolsep) * \real{0.3000}}@{}}
\caption{Grid Table example}\tabularnewline
\toprule\noalign{}
\begin{minipage}[b]{\linewidth}\raggedright
Col1
\end{minipage} & \begin{minipage}[b]{\linewidth}\raggedright
Col2
\end{minipage} & \begin{minipage}[b]{\linewidth}\raggedright
Col3
\end{minipage} & \begin{minipage}[b]{\linewidth}\raggedright
Col4
\end{minipage} & \begin{minipage}[b]{\linewidth}\raggedright
Col5
\end{minipage} \\
\midrule\noalign{}
\endfirsthead
\toprule\noalign{}
\begin{minipage}[b]{\linewidth}\raggedright
Col1
\end{minipage} & \begin{minipage}[b]{\linewidth}\raggedright
Col2
\end{minipage} & \begin{minipage}[b]{\linewidth}\raggedright
Col3
\end{minipage} & \begin{minipage}[b]{\linewidth}\raggedright
Col4
\end{minipage} & \begin{minipage}[b]{\linewidth}\raggedright
Col5
\end{minipage} \\
\midrule\noalign{}
\endhead
\bottomrule\noalign{}
\endlastfoot
a & b & c & d & e \\
d & & & & \\
\end{longtable}

\subsection{Complex grid table
example}\label{complex-grid-table-example}

This table was generated by tt() as the output of an r chunk in a Quarto
doc. For larger multi-page tables, this method gives correct page breaks
in the pdf and html outputs. You can change the relative column widths
with \{tbl-colwidths=``{[}10,20,20,20,30{]}''\} placed after the table
caption declaration at the end.

\begin{longtable}[]{@{}
  >{\raggedright\arraybackslash}p{(\columnwidth - 8\tabcolsep) * \real{0.1000}}
  >{\raggedright\arraybackslash}p{(\columnwidth - 8\tabcolsep) * \real{0.2000}}
  >{\raggedright\arraybackslash}p{(\columnwidth - 8\tabcolsep) * \real{0.2000}}
  >{\raggedright\arraybackslash}p{(\columnwidth - 8\tabcolsep) * \real{0.2000}}
  >{\raggedright\arraybackslash}p{(\columnwidth - 8\tabcolsep) * \real{0.3000}}@{}}
\caption{More complex Grid Table example}\tabularnewline
\toprule\noalign{}
\begin{minipage}[b]{\linewidth}\raggedright
Reagent type (species) or resource
\end{minipage} & \begin{minipage}[b]{\linewidth}\raggedright
Designation
\end{minipage} & \begin{minipage}[b]{\linewidth}\raggedright
Source or reference
\end{minipage} & \begin{minipage}[b]{\linewidth}\raggedright
Identifiers
\end{minipage} & \begin{minipage}[b]{\linewidth}\raggedright
Additional information
\end{minipage} \\
\midrule\noalign{}
\endfirsthead
\toprule\noalign{}
\begin{minipage}[b]{\linewidth}\raggedright
Reagent type (species) or resource
\end{minipage} & \begin{minipage}[b]{\linewidth}\raggedright
Designation
\end{minipage} & \begin{minipage}[b]{\linewidth}\raggedright
Source or reference
\end{minipage} & \begin{minipage}[b]{\linewidth}\raggedright
Identifiers
\end{minipage} & \begin{minipage}[b]{\linewidth}\raggedright
Additional information
\end{minipage} \\
\midrule\noalign{}
\endhead
\bottomrule\noalign{}
\endlastfoot
biological sample (N. vectensis) & larval, juvenile and adult N.
vectensis & Specimens obtained form the Marine Invertebrate Culture Unit
of the University of Exeter & N/A & NA \\
biological sample (cDNA) & cDNA obtained from N. vectensis & this study
& N/A & RNA extracted with Trizol and cDNA synthesized with cDNA
synthesis kit according to manufacturers recommendation \\
biological sample (peptide extract) & peptide extracts obtained from N.
vectensis & this study & N/A & Peptides extracted from N. vectensis
according to protocol explained in Material and Methods \\
genetic reagent (cDNA synthesis) & SuperScript™ III First-Strand
Synthesis System & Invitrogen (from ThermoFisher) & 18080051 & NA \\
genetic reagent (Polymerase) & Q5® Hot Start High-Fidelity DNA
Polymerase & New England Biolabs & M0493L & NA \\
genetic reagent (DNA assembly) & NEBuilder® HiFi DNA Assembly Master Mix
& New England Biolabs & E2621L & NA \\
genetic reagent (restriction enzyme) & EcoRV restriction enzyme & New
England Biolabs & R3195L & NA \\
genetic reagent (restriction enzyme) & Afl2 restriction enzyme & New
England Biolabs & R0520L & NA \\
\end{longtable}

\subsection{References}\label{references}

\phantomsection\label{refs}
\begin{CSLReferences}{1}{0}
\bibitem[\citeproctext]{ref-Jacobs2023}
Jacobs EAK, Ryu S. 2023. Larval zebrafish as a model for studying
individual variability in translational neuroscience research.
\emph{Frontiers in Behavioral Neuroscience} \textbf{17}.
doi:\href{https://doi.org/10.3389/fnbeh.2023.1143391}{10.3389/fnbeh.2023.1143391}

\bibitem[\citeproctext]{ref-Jokura_2023}
Jokura K, Ueda N, Gühmann M, Yañez-Guerra LA, Słowiński P, Wedgwood KCA,
Jékely G. 2023. Nitric oxide feedback to ciliary photoreceptor cells
gates a UV avoidance circuit.
doi:\href{https://doi.org/10.7554/elife.91258.1}{10.7554/elife.91258.1}

\bibitem[\citeproctext]{ref-Marinkovi2020}
Marinković M, Berger J, Jékely G. 2019. Neuronal coordination of motile
cilia in locomotion and feeding. \emph{Philosophical Transactions of the
Royal Society B: Biological Sciences} \textbf{375}:20190165.
doi:\href{https://doi.org/10.1098/rstb.2019.0165}{10.1098/rstb.2019.0165}

\bibitem[\citeproctext]{ref-ozpolat2021}
Ozpolat BD, Randel N, Williams EA, Bezares-Calderón LA, Andreatta G,
Balavoine G, Bertucci PY, Ferrier DEK, Gambi MC, Gazave E,
Handberg-Thorsager M, Hardege J, Hird C, Hsieh Y-W, Hui J, Mutemi KN,
Schneider SQ, Simakov O, Vergara HM, Vervoort M, Jékely G,
Tessmar-Raible K, Raible F, Arendt D. 2021. The Nereid on the rise:
Platynereis as a model system. \emph{Zenodo}.
doi:\href{https://doi.org/10.5281/ZENODO.4907400}{10.5281/ZENODO.4907400}

\end{CSLReferences}



\end{document}
